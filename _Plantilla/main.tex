\documentclass{replab}
\usepackage{lipsum}
\usepackage{float}
\usepackage{biblatex}

% --- Información del documento ---
\title{Laboratorio de Electrónica}
\author{Alarcón Pérez Bruno}

\date{\today}
\subtitle={Práctica X}
\email={\href{mailto:bruno_alarcon@ciencias.unam.mx}{\color{principaluno}\texttt{bruno\_alarcon@ciencias.unam.mx}}}
\subject={Laboratorio de Electrónica}

\setlength{\columnsep}{14pt}

% --- Archivo de bibliografía ---
\addbibresource{repbib.bib}

% --- Inicio del documento ---
\begin{document}
	
	\pagestyle{fancy}
	\unspacedoperators
	
% --- Título ---
	\twocolumn[
		\begin{center}
			\maketitle
				
			{\begin{tcolorbox}[colframe=white, colback=principaldos, arc=8pt]
				\begin{onecolabstract}
					Lo que se hizo, lo que se buscaba observar, lo que se observó
					y conlcusión final.

					\medskip
				\end{onecolabstract}

			\end{tcolorbox}}

			\smallskip
		\end{center}
	]
	
% --- Cuerpo del reporte ---
	
	\section{Montaje}

    Incluir el esquema del circuito, si es necesaria alguna foto del mismo
	y añadir una breve explicación de este.

	\section{Mediciones y observaciones}

    Tablas, anotaciones interesantes de lo observado ya sea en el montaje o en las mediciones.

	\section{Resultados y Discución}

    Gráficas, comparaciones y análisis de esto; coincidencias y/o discrepancias con la teoría,
	posibles errores.

	\section{Conclusión}

    Breve texto del aprendizaje final.

	% \printbibliography

	\appendix
    \section{Primer apéndice}

    APÉNDICES
	
\end{document}