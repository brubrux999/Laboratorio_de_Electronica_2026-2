\documentclass{replab}
\usepackage{lipsum}
\usepackage{float}
\usepackage{biblatex}

% --- Información del documento ---
\title{Laboratorio de Electrónica}
\author{Alarcón Pérez Bruno}

\date{\today}
\subtitle={Práctica 1: LEY DE OHM}
\email={\href{mailto:bruno_alarcon@ciencias.unam.mx}{\color{principaluno}\texttt{bruno\_alarcon@ciencias.unam.mx}}}
\subject={Laboratorio de Electrónica}

\setlength{\columnsep}{14pt}

% --- Archivo de bibliografía ---
\addbibresource{repbib.bib}

% --- Inicio del documento ---
\begin{document}
	
	\pagestyle{fancy}
	\unspacedoperators
	
% --- Título ---
	\twocolumn[
		\begin{center}
			\maketitle
				
			{\begin{tcolorbox}[colframe=white, colback=principaldos, arc=8pt]
				\begin{onecolabstract}
					Mediante la aplicación del método de Maxwell para calcular el voltaje y corriente en cada resistencia de un circuito,
					se buscó observar un comportamiento similar en el laboratorio mediante el armado del mismo circuito en una protoboard,
					aprendiendo a medir estos valores cuidando a la vez los instrumentos.

					\medskip
				\end{onecolabstract}

			\end{tcolorbox}}

			\smallskip
		\end{center}
	]
	
% --- Cuerpo del reporte ---
	
	\section{Montaje}

    Se armó un circuito en una protoboard en el cual se conectaron una fuente DC y 11 resistencias
	de dos valores distintos aproximadamente; $560\Omega$ y $820\Omega$, como lo muestra el esquema siguiente.

	\begin{figure}[!ht]
		\centering
		\includegraphics[width=0.45\textwidth]{circuitos/Circuito 1.jpg}
		\caption{Esquema del circuito armado en la protoboard}
		\label{fig:circuito1}
	\end{figure}

	Una analogía a un circuito eléctrico sencillo como el mostrado, podría ser una corriente de aire en un lugar cerrado con ventanas; la fuente
	de voltaje podría ser un ventilador, una ventana o puerta abierta, los pasillos y salas del lugar es el material conductor donde fluye la corriente
	y las resistencias podrían ser las ventanas y puertas las cuales su valor de resistencia aumenta al estar más cerradas.

	\section{Mediciones y observaciones}

    Antes de pasar a las mediciones en el circuito, se realizaron los calculos teóricos para los valores
	de voltaje, corriente y potencia para cada resistencia considerando un voltaje en la fuente de $10 V$. El desarrollo se encuentra en
	el Apéndice \ref{apx:1}.

	Para la medicón de los valores mencionados, se usó un multímetro digital y un par de cables con conexión banana-caiman. Es importante mencionar
	que, antes de encender la fuente DC se debe asegurar que la perilla del voltaje este al mínimo mientra que la de corriente al máximo, posteriormente,
	se colocó un voltaje de $10 V$.

	Primero se midió la resistencia de cada resistencia para corroborar su valor, posteriormente, para el voltaje, la conexión a cada resistencia
	fue en paralelo, es decir, en cada ``patita'' de estas, en cambio, la corriente se midió en serie de modo que se abrió el circuito, ya sea
	del lado izquierdo o derecho de la resistencia, para incluir el multímetro en este. Finalmente la potencia se obtuvo multiplicando el voltaje
	por la corriente.\\
	Los resultados se muestran en la Tabla \ref{tab:circuito1}.

	\begin{table}[!ht]
		\centering
		\begin{tabular}{|c|c|c|c|c|}
		\hline
			& $\Omega$ & V [$V$] & I [$mA$] & P [$mW$] \\ \hline
		R1  & 556   & 3.4         & 6.1      & 20.7         \\ \hline
		R2  & 822   & 3.3         & 4.1      & 13.5         \\ \hline
		R3  & 560   & 3.4         & 6.1      & 20.7         \\ \hline
		R4  & 818   & 1.7         & 2.1      & 3.6          \\ \hline
		R5  & 557   & 0.9         & 1.6      & 1.4          \\ \hline
		R6  & 823   & 0.7         & 0.9      & 0.6          \\ \hline
		R7  & 556   & 0.7         & 1.3      & 0.9          \\ \hline
		R8  & 818   & 0.4         & 0.5      & 0.2          \\ \hline
		R9  & 562   & 0.3         & 0.5      & 0.2          \\ \hline
		R10 & 821   & 0.2         & 0.2      & 0.04         \\ \hline
		R11 & 555   & 0.2         & 0.3      & 0.06         \\ \hline
		\end{tabular}
		\caption{
			Valores medidos de resistencia, voltaje, corriente y
			potencia en cada resistencia, respectivamente, del circuito mostrado en el Esquema \ref{fig:circuito1}.
		}
		\label{tab:circuito1}
	\end{table}

	\section{Resultados y Discución}
 
	Podemos observar que, aunque se usaron distintos valores de resistencias para los calculos teóricos y las mediciones experimentales,
	se observa claramente una tendencia en la disminución de la potencia de las resistencias a medida que se alejan de la fuente. Asimismo,
	se obtuvieron valores casi idénticos en aquellas resistencias que solo se encontraban en una sola malla del circuito, como $R1$ y $R3$ o $R8$ y $R9$.

	\section{Conclusión}

    La medición del voltaje y corriente y el cálculo de la potencia para cada resistencia de un circuito resultó ser semejante a los cálculos teóricos
	del mismo circuito. Se aprendió que el voltaje se mide en paralelo mientras que la corriente en serie, así como que la fuente DC se debe encender
	con el voltaje al mínimo y la corriente al máximo, de este modo se cuida de manera correcta los intrumentos del laboratorio.

	% \printbibliography

	\onecolumn
	\appendix
    \section{Cálculo de valores teóricos}
	\label{apx:1}

    \begin{figure}[!ht]
		\centering
		\begin{subfigure}[b]{0.40\linewidth}
			\includegraphics[width=\linewidth]{imagenes/20260215_150617.jpg}
		\end{subfigure}
		\hspace{0.2mm}
		\begin{subfigure}[b]{0.40\linewidth}
			\includegraphics[width=\linewidth]{imagenes/20260215_150701.jpg}
		\end{subfigure}

		\begin{subfigure}[b]{0.40\linewidth}
			\includegraphics[width=\linewidth]{imagenes/20260215_150729.jpg}
		\end{subfigure}
		\hspace{0.2mm}
		\begin{subfigure}[b]{0.40\linewidth}
			\includegraphics[width=\linewidth]{imagenes/20260215_150850.jpg}
		\end{subfigure}
		
		\caption{
			Desarrollo del método de Maxwell para el cálculo del voltaje,
			corriente y potencia de cada resistencia.
		}
		\label{fig:mawell1}
	\end{figure}
	
\end{document}